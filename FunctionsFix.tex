\section{Function}
\subsection{Pengertian Function}
Function atau Fungsi adalah kode terorganisir dan dapat digunakan kembali yang digunakan untuk melakukan tindakan tunggal dan terkait. 
Fungsi menyediakan modularitas yang lebih baik untuk aplikasi Anda dan tingkat penggunaan kode yang tinggi. Seperti yang sudah Anda ketahui, 
Python memberi Anda banyak fungsi built-in seperti cetak (), dll. Tetapi Anda juga dapat membuat fungsi Anda sendiri. 
Fungsi ini disebut fungsi yang ditentukan pengguna.

Anda dapat menentukan fungsi untuk menyediakan fungsionalitas yang dibutuhkan. Berikut adalah aturan sederhana untuk mendefinisikan fungsi dengan Python. 
Fungsi (function) adalah yang bertugas untuk mencari akar dari suatu angka. Secara umum yang dimaksud dengan function adalah statemen
yang dieksekusi. 


Fungsi (Function) adalah suatu program terpisah dalam blok sendiri yang berfungsi sebagai sub-program (modul program) yang merupakan sebuah program kecil untuk memproses sebagian dari pekerjaan program utama.

\subsection{Keuntungan Function}
Keuntungan menggunakan fungsi :
\begin{enumerate}
\item
Program besar dapat di pisah-pisah menjadi program-program kecil melalui function.
\item
Kemudahan dalam mencari kesalahan-kesalahan karena alur logika jelas dan kesalahan
dapat dilokalisasi dalam suatu modul tertentu.
\item
Memperbaiki atau memodifikasi program dapat dilakukan pada suatu modul tertentu saja
tanpa menggangu keseluruhan program.
\item
Dapat digunakan kembali (Reusability) oleh program atau fungsi lain.
\item
Meminimalkan penulisan perintah yang sama.
\end{enumerate}

\subsection{Kategori Function}
Kategori Fungsi
\begin{enumerate}
\item
Standard Library Function
adalah fungsi-fungsi yang telah disediakan oleh Interpreter Python dalam file-file atau librarynya.
 Misalnya: raw\_input(), input(), print(), open(), len(), max(), min(), abs() dll.
\item
Programme-Defined Function
Adalah function yang dibuat oleh programmer sendiri. Function ini memiliki nama tertentu yang
unik dalam program, letaknya terpisah dari program utama, dan bisa dijadikan satu ke dalam suatu
library buatan programmer itu sendiri.
\end{enumerate}

\subsection{Perintah pada Function}
Dalam python terdapat dua perintah yang dapat digunakan untuk membuat sebuah fungsi,
yaitu def dan lambda. def adalah perintah standar dalam python untuk mendefinisikan sebuah
fungsi. Tidak seperti function dalam bahasa pemrograman compiler seperti C/C++, def dalam
python merupakan perintah yang executable, artinya function tidak akan aktif sampai python merunning
perintah def tersebut. Sedangkan lambda, dalam python lebih dikenal dengan nama
Anonymous Function (Fungsi yang tidak disebutkan namanya). Lambda bukanlah sebuah perintah
(statemen) namun lebih kepada ekspresi (expression).

Blok fungsi dimulai dengan defensi kata kunci diikuti oleh nama fungsi dan tanda kurung (()). 
Setiap parameter masukan atau argumen harus ditempatkan di dalam tanda kurung ini. Anda juga dapat menentukan parameter di dalam tanda kurung ini. 
Pernyataan fungsi pertama dapat berupa pernyataan opsional - string dokumentasi fungsi atau docstring. 
Blok kode dalam setiap fungsi dimulai dengan titik dua (:) dan indentasi. 

Pernyataan kembali [ekspresi] keluar dari sebuah fungsi, secara opsional menyampaikan kembali ekspresi ke pemanggil. Pernyataan pengembalian tanpa argumen sama dengan return None.

Secara default, parameter memiliki perilaku posisi dan Anda perlu memberi tahu mereka dengan urutan yang sama seperti yang ditetapkan.

Fungsi berikut mengambil string sebagai parameter masukan dan mencetaknya di layar standar.
Mendefinisikan sebuah fungsi hanya memberinya sebuah nama, menentukan parameter yang akan disertakan dalam fungsi dan menyusun blok kode. Setelah struktur dasar fungsi selesai, Anda dapat menjalankannya dengan memanggilnya dari fungsi lain atau langsung dari prompt Python. Berikut adalah contoh untuk memanggil fungsi printme ()

Bila kode diatas dieksekusi, maka menghasilkan hasil sebagai berikut 

Semua parameter (argumen) dalam bahasa Python dilewatkan dengan referensi. Ini berarti jika Anda mengubah parameter yang mengacu pada suatu fungsi, perubahan tersebut juga mencerminkan kembali fungsi pemanggilan. 

\subsection{Cara Memanggil Function}
Anda dapat memanggil fungsi dengan menggunakan jenis argumen formal berikut: 
\begin{itemize}
\item
Argumen yang dibutuhkan 
\item
Argumen kata kunci 
\item
Argumen baku 
\item
Argumen panjang variable 
\end{itemize}

\subsection{Contoh Fungsi}
\begin{verbatim}
\# lat15.py

def halo_dunia():
    print 'Halo Dunia!'
halo_dunia()  \# memanggil fungsi halo_dunia
halo_dunia()  \# fungsi halo_dunia dipanggil lagi
\end{verbatim}

\subsection{Contoh Parameter Fungsi}
\begin{verbatim}
def halo(nama):
    print 'Halo \%s!' \% nama

def cetak_maksimal(a, b):
    if a > b:
        print '\%s merupakan nilai maksimal' \% a
    elif a == b:
        print '\%s sama dengan \%s' \% (a, b)
    else:
        print '\%s merupakan nilai maksimal' \% b

halo('Dunia')  \# memanggil fungsi halo dengan argumen 'Dunia'
halo('Indonesia')  \# memanggil fungsi halo dengan argumen 'Indonesia'

cetak_maksimal(10, 100)

x = 9
y = 3

cetak_maksimal(x, y)

\end{verbatim}



Argumen yang diperlukan adalah argumen yang diberikan ke sebuah fungsi dalam urutan posisi yang benar. Di sini, jumlah argumen dalam pemanggilan fungsi harus sesuai persis dengan definisi fungsi. Untuk memanggil fungsi printme (), Anda pasti perlu melewati satu argumen, jika tidak maka akan memberikan kesalahan sintaks sebagai berikut 

Argumen kata kunci terkait dengan pemanggilan fungsi. Bila Anda menggunakan argumen kata kunci dalam pemanggilan fungsi, penelepon mengidentifikasi argumen berdasarkan nama parameter. Hal ini memungkinkan Anda melewatkan argumen atau menempatkannya agar tidak bermasalah karena penerjemah Python dapat menggunakan kata kunci yang diberikan agar sesuai dengan nilai parameter. Anda juga dapat membuat panggilan kata kunci ke fungsi printme () dengan cara berikut 

Bila kode diatas dieksekusi, maka menghasilkan hasil sebagai berikut 

Argumen default adalah argumen yang mengasumsikan nilai default jika nilai tidak diberikan dalam pemanggilan fungsi untuk argumen itu. Contoh berikut memberi ide pada argumen default, ini mencetak usia default jika tidak lulus

Bila kode diatas dieksekusi, maka menghasilkan hasil sebagai berikut 

Anda mungkin perlu memproses sebuah fungsi untuk argumen lebih banyak daripada yang Anda tentukan saat menentukan fungsinya. Argumen ini disebut variable-lengtharguments dan tidak disebutkan dalam definisi fungsi, tidak seperti argumen yang dibutuhkan dan standar. 
Sintaks untuk fungsi dengan argumen variabel non-kata kunci adalah ini 

Tanda asterisk (*) ditempatkan sebelum nama variabel yang menyimpan nilai dari semua argumen variabel nonkeyword. Tuple ini tetap kosong jika tidak ada argumen tambahan yang ditentukan selama pemanggilan fungsi. Berikut adalah contoh sederhana

Fungsi ini disebut anonim karena tidak dinyatakan secara standar dengan menggunakan kata kunci def. Anda bisa menggunakan kata kunci lambda untuk membuat fungsi anonim yang kecil. \par

Bentuk lambda bisa mengambil sejumlah argumen tapi hanya mengembalikan satu nilai dalam bentuk ekspresi. Mereka tidak dapat berisi perintah atau beberapa ekspresi. \par
Fungsi anonim tidak bisa menjadi panggilan langsung untuk dicetak karena lambda  membutuhkan ekspresi Fungsi Lambda memiliki namespace lokal mereka sendiri dan tidak dapat mengakses variabel selain yang ada dalam daftar parameter dan yang ada di namespace global. 
Meskipun tampak bahwa lambda adalah versi satu baris dari sebuah fungsi, mereka tidak setara dengan pernyataan inline di C atau C ++, yang tujuannya adalah dengan  melewatkan alokasi stack fungsi selama pemanggilan untuk alasan kinerja. 
Sintaks fungsi lambda hanya berisi satu pernyataan, yaitu sebagai berikut 

Pernyataan kembali [ekspresi] keluar dari sebuah fungsi, secara opsional menyampaikan kembali ekspresi ke pemanggil. Pernyataan pengembalian tanpa argumen sama dengan return None. 

Semua contoh di atas tidak mengembalikan nilai apapun. Anda bisa mengembalikan nilai dari sebuah fungsi sebagai berikut

Semua variabel dalam sebuah program mungkin tidak dapat diakses di semua lokasi dalam program tersebut. Ini tergantung di mana Anda telah menyatakan sebuah variabel. 
 
Ruang lingkup variabel menentukan bagian dari program di mana Anda dapat mengakses pengenal tertentu. Ada dua lingkup dasar variabel dengan Python 

Variabel yang didefinisikan di dalam badan fungsi memiliki lingkup lokal, dan yang didefinisikan di luar memiliki cakupan global. 

Ini berarti bahwa variabel lokal dapat diakses hanya di dalam fungsi di mana mereka dideklarasikan, sedangkan variabel global dapat diakses di seluruh tubuh program oleh semua fungsi. Saat Anda memanggil fungsi, variabel yang dideklarasikan di dalamnya dibawa ke lingkup. Berikut adalah contoh sederhana 

Dalam   menulis   program, tentunya   kita   akan menggunakan kode program secara efesien, source code yang pernah kita tulis sebelumnya, pastilah akan kita  
gunakan  kembali,  dengan  beberapa  nilai  yang  berbeda. 
Tentu saja kita tidak mungkin menuliskan kembali  kode  yang  ingin  dipanggil  ulang  tersebut.  
Solusinya,  kita  dapat  mengelompokkan  kode-kode  yang sering dipanggil ulang dalam suatu kelompok kode yang disebut fungsi. 
Selain  itu  juga  Anda  dapat  memecah  masalah  besar  menjadi  masalah-masalah  yang  lebih  kecil.  Dalam  C  atau  bahasa  pemrograman  lain,  biasanya  digunakan istilah function. 

Beberapa hal yang perlu diperhatikan terkait dengan penggunaan fungsi :
Deklarasi fungsi, digunakan kata kunci def, berguna untuk membuat obyek fungsi dan selanjutnya melakukan assignment obyek fungsi tersebut dengan sebuah nama.
Pada istilah passing parameters by preference dn passing parameter by value, maka bahasa pemrograman Python melakukan passing parameter
by assignment.
Sepertivariabel, kita perlu mendeklarasikan tipenya terlebih dahulu sehingga parameter pada fungsi bisa digunakan untuk berbagai tipe obyek yang sesuai.
Bentuk umum :
def fungsi(argumen1, argumen2, ..., argumen n):
 Statement1
 ...
 Statement n
 return return variable
Keterangan : 
Fungsi diawali dengan kata kunci def, diikuti nama fungsi, boleh diikuti parameter formal yang ditulis dalam tanda kurung, dan deklarasi fungsi ditutup dengan tanda titik dua (:).
