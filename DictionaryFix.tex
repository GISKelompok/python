\section{Dictionary}
\subsection{Pengertian Dictionary}
Dictionary merupakan tipe berupa kumpulan data seperti list dan tuple namun disajikan dalam bentuk yang tidak berurutan. Mungkin bagi bahasa lain penggunaan 
dictionary sangat aneh dan tidak ada. Setiap Elemen data dalam dictionary dibagi menjadi key dan value. Bagi para programmer yang terbiasa dengan penggunaan seleksi case, akan kebingungan saat mengenal 
python karena tidak tersedia. Namun dengan adanya dictionary permasalahan tersebut dapat dipecahkan.

Dictionary Python berbeda dengan List ataupun Tuple. Karena setiap urutanya berisi key dan value. Setiap key dipisahkan dari value-nya oleh titik dua (:), item dipisahkan oleh koma, dan semuanya tertutup dalam kurung kurawal. Dictionary kosong tanpa barang ditulis hanya dengan dua kurung kurawal, seperti ini: {}.

Dictionary seperti buku alamat, dengan buku alamat anda bisa mencari alamat atau detail kontak hanya menggunakan nama orang yang anda cari. Kita mengasosiasikan key (nama) dengan value (detail). Catatan key harus bersifat unik, anda tidak bisa menemukan informasi yang tepat jika ada dua orang yang mempunyai nama yang sama dalam buku alamat anda.

Anda hanya bisa menggunakan obyek immutable (seperti string) untuk key/ kunci dictionary. Anda bisa menggunakan obyek mutable atau immutable untuk value dalam dictionary.

Dictionary dispesifikasikan menggunakan pasangan key dan value diapit menggunakan kurung kurawal, {key1: value1, key2: value2}.


Dictionary mencakup setiap data akan memiliki pengenalnya masing – masing. Pengenal tersebut dinamakan dengan key dan datanya dinamakan dengan value.
Dictionary diawali dengan tanda '{' dan diakhiri dengan tanda '}'. Khusus untuk key pada dictionary, nilainya harus berupa tipe data yang tidak dapat diganti seperti tuple, string dan number. Tapi umumnya key berisi number dan string. Karena jika Anda mencoba memasukkan tipe data yang mutable, akan keluar peringatan 'unhashable type' saat mendefinisikan dictionary yang key-nya
berupa tipe data mutable.

\subsection{Contoh Dictionary}

\begin{verbatim}
data = {"nama":"ganjar","umur":30,"gender":"perempuan"}
print data
print data['nama']
print data['umur']

#via loop
for key in data:
    print key,"==>", data[key]
\end{verbatim}


\subsubsection{Membuat Dictionary}
biodata1 = {'Nama': 'Wayan', 'Asal': 'Gianyar', 'Umur': 21, 'NoUrut': 1};
biodata2 = {'Nama': 'Made', 'Asal': 'Denpasar', 'Umur': 23, 'NoUrut': 2};
biodata3 = {'Nama': 'Nyoman', 'Asal': 'Gianyar', 'Umur': 21, 'NoUrut': 3};
biodata4 = {'Nama': 'Wayan', 'Asal': 'Gianyar', 'Umur': 21, 'NoUrut': 1};
 
\subsubsection{Membandingkan Dictionary}
print "Hasil perbandingan biodata1 dan biodata2 : %d" % cmp (biodata1, biodata2)
print "Hasil perbandingan biodata1 dan biodata3 : %d" % cmp (biodata1, biodata3)
print "Hasil perbandingan biodata1 dan biodata4 : %d" % cmp (biodata1, biodata4)
 
\subsubsection{Mendapatkan Jumlah Item  Dictionary}
print "Panjang biodata1 : %d" % len (biodata1)

Dictionary mencakup setiap data akan memiliki pengenalnya masing – masing. Pengenal tersebut dinamakan dengan key dan datanya dinamakan dengan value.
Dictionary diawali dengan tanda '{' dan diakhiri dengan tanda '}'. Khusus untuk key pada dictionary, nilainya harus berupa tipe data yang tidak dapat diganti seperti tuple, string dan number. Tapi umumnya key berisi number dan string. Karena jika Anda mencoba memasukkan tipe data yang mutable, akan keluar peringatan 'unhashable type' saat mendefinisikan dictionary yang key-nya
berupa tipe data mutable.

Kunci unik dalam kamus sementara nilai mungkin tidak. Nilai kamus bisa berupa tipe apa pun, namun kunci harus berupa tipe data yang tidak berubah seperti string, angka, atau tupel. 

Untuk mengakses elemen kamus, Anda dapat menggunakan tanda kurung siku yang sudah dikenal bersama dengan kunci untuk mendapatkan nilainya. Berikut adalah contoh sederhana 

Jika kita mencoba mengakses item data dengan sebuah kunci, yang bukan bagian dari kamus, kita mendapatkan error sebagai berikut 

Bila kode diatas dieksekusi, maka menghasilkan hasil sebagai berikut 

Anda dapat memperbarui kamus dengan menambahkan entri baru atau pasangan nilai kunci, memodifikasi entri yang ada, atau menghapus entri yang ada seperti yang ditunjukkan di bawah ini dalam contoh sederhana 

Bila kode diatas dieksekusi, maka menghasilkan hasil sebagai berikut

Anda dapat menghapus elemen kamus individual atau menghapus keseluruhan isi kamus. Anda juga dapat menghapus seluruh kamus dalam satu operasi.

Untuk menghapus seluruh kamus secara eksplisit, cukup gunakan del statement. Berikut adalah contoh sederhana

Ini menghasilkan hasil berikut. Perhatikan bahwa pengecualian diajukan karena setelah kamus del dict tidak ada lagi metode dibahas di bagian selanjutnya. \par

Nilai kamus tidak memiliki batasan. Mereka bisa menjadi objek Python yang sewenang-wenang, baik objek standar atau objek yang ditentukan pengguna. Namun, hal yang sama tidak berlaku untuk kunci. 
 
Ada dua hal penting yang perlu diingat tentang kunci kamus 
Lebih dari satu entri per kunci tidak diperbolehkan. Yang berarti tidak ada kunci duplikat yang diperbolehkan. Ketika kunci duplikat ditemui selama penugasan, tugas terakhir akan menang. 
Membuat kamus sama mudahnya dengan menempatkan item dalam kurung kurawal  \$  \{  \$ \$  \}  \$ dipisahkan dengan koma. Item memiliki kunci dan nilai yang sesuai dinyatakan sebagai pasangan, kunci: nilai. Sementara nilai dapat berupa tipe data apa pun dan dapat diulang, kunci harus terdiri dari tipe yang tidak dapat diubah (string, number atau tupel dengan elemen yang tidak berubah) dan harus unik.

Seperti yang bisa Anda lihat di atas, kita juga bisa membuat kamus menggunakan fungsi built-in dict (). 

Sementara pengindeksan digunakan dengan jenis wadah lain untuk mengakses nilai, kamus menggunakan tombol. Kunci dapat digunakan baik di dalam tanda kurung siku atau dengan metode get (). Perbedaan saat menggunakan get () adalah mengembalikan Elemen alih-alih KeyError, jika kuncinya tidak ditemukan. \par

Saat menjalankan program, hasilnya adalah: 

Kamus bisa berubah-ubah. Kita bisa menambahkan item baru atau mengubah nilai barang yang ada menggunakan operator penugasan. 
Jika kuncinya sudah ada, nilai akan diperbarui, jika ada kunci baru: pasangan nilai ditambahkan ke kamus. 

Saat menjalankan program, hasilnya adalah: 

Kita bisa menghapus item tertentu dalam kamus dengan menggunakan metode pop (). Metode ini menghilangkan item dengan tombol yang disediakan dan mengembalikan nilainya. \par

Metodenya, popitem () dapat digunakan untuk menghapus dan mengembalikan item yang sewenang-wenang (key, value) membentuk kamus. Semua item dapat dihapus sekaligus dengan menggunakan metode clear (). \par

Kita juga bisa menggunakan kata kunci del untuk menghapus setiap item atau keseluruhan kamus itu sendiri. 

Tipe data daftar memiliki beberapa metode lagi. Berikut adalah semua metode daftar objek: 

Tambahkan item ke bagian akhir daftar; setara dengan [len (a):] = [x]. 

Perluas daftar dengan menambahkan semua item dalam daftar yang diberikan; setara dengan [len (a):] = L. 
 
Masukkan item pada posisi tertentu. Argumen pertama adalah indeks dari elemen yang sebelum dimasukkan, jadi a.insert (0, x) memasukkan di bagian depan daftar, dan a.insert (len (a), x) setara dengan a.append ( x). 
Hapus item pertama dari daftar yang nilainya x. Ini adalah kesalahan jika tidak ada item seperti itu. 
 
Hapus item pada posisi yang diberikan dalam daftar, dan kembalikan. Jika tidak ada indeks yang ditentukan, a.pop () menghapus dan mengembalikan item terakhir dalam daftar. (Tanda kurung siku di sekitar i pada tanda tangan metode menunjukkan bahwa parameternya adalah opsional, bukankah Anda harus mengetikkan tanda kurung siku pada posisi itu. Anda akan sering melihat notasi ini di Referensi Perpustakaan Python.) \par

Kembalikan indeks di daftar item pertama yang nilainya x. Ini adalah kesalahan jika tidak ada item seperti itu. 
 
Kembalikan berapa kali x muncul dalam daftar.

Urutkan item daftar di tempat (argumen dapat digunakan untuk kustomisasi sortir, lihat diurutkan () untuk penjelasan mereka).

Membalik unsur daftar, di tempat. 

